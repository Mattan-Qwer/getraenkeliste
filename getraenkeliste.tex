\documentclass[paper=a4, twoside=false, fontsize=14pt, headings=normal]{article} % DIN-A4 Formatierung
%-- Pakete --%
\usepackage[ngerman]{babel} %Ausgabe von Titeln in Deutsch
\usepackage[utf8]{inputenc} %Umlaute
% Mathebibliotheken
\usepackage{amsmath}
\usepackage{amssymb}
\usepackage{amsthm}
\usepackage{esint}
\usepackage{esvect}
% bessere Schriftarten
\usepackage{mathptmx, charter, courier}
\usepackage[scaled]{helvet}
\usepackage{lmodern}
\renewcommand*\familydefault{\sfdefault}
% Schriftverbesserung, microtype
\usepackage{microtype}
% farbige Schrift
\usepackage{color, colortbl}
\usepackage[table]{xcolor}
\definecolor{lightgray}{gray}{0.9}
% Bilder
\usepackage{graphicx}
% URLs
\usepackage{hyperref}
%-- Pagestyle --%
% Zeilenabstand
\usepackage{setspace}
% Zeichenabstand
\usepackage{xspace}
%-- Tabellen --%
\usepackage{spreadtab}
\usepackage{numprint}
\usepackage{eurosym}
\usepackage{fp}
% Seitenränder
\usepackage[left=1.5cm,right=1.5cm,top=.5cm,bottom=.5cm]{geometry}
% Keine Seitenzahlen usw.
\pagestyle{empty}

\begin{document}

% Formatierung
\onehalfspacing
\begin{center}

% Um Logo in Zeile vertikal zu zentrieren
\raisebox{-.25\height}{\includegraphics[width=0.3\textwidth]{logo}} \huge \textbf{Getränkestrichliste}

% Formatierung - normale Schrift
\vfill
\large

% Optionen
\renewcommand\STprintnum[1]{\FPifneg{#1}\color{red}\fi\numprint{#1}}
\npdecimalsign{,}
\npthousandsep{.}


% hline-Dicke und Zellenfarbe
\setlength{\arrayrulewidth}{.09em}

% Getränkeliste
\rowcolors{1}{}{lightgray}
\begin{spreadtab}{{tabular}{|l|r|p{3.2cm}|p{3.2cm}|p{3.2cm}|c|}}
\hline
@\textbf{Name} 		&@\textbf{Guthaben}  									&@\textbf{\EUR{0,50}}	&@\textbf{\EUR{0,70}} 	&@\textbf{\EUR{0,80}} 	&@\textbf{Score} 			\\
\hline
%
@ John Doe 	& \nprounddigits{2} :={23 - 0.5*[1,0] - 0.7*[2,0] - 0.8*[3,0]} \EUR{} \npnoround 	&~\color{gray}:={5}~	&~\color{gray}:={11}~	&~\color{gray}:={1}~	&[-3,0] + [-2,0] + [-1,0] 	\\\hline
@ Jane Doe 	& \nprounddigits{2} :={5 - 0.5*[1,0] - 0.7*[2,0] - 0.8*[3,0]} \EUR{} \npnoround 	&~\color{gray}:={15}~	&~\color{gray}:={11}~	&~\color{gray}:={1}~	&[-3,0] + [-2,0] + [-1,0] 	\\\hline
% HIER NEUE EINFÜGEN
%
\hline
@ $\sum$ 			&  :={sum(b2:[0,-1])} \EUR{}	 						& sum(c2:[0,-1]) 		& sum(d2:[0,-1]) 		& sum(e2:[0,-1])  				&[-3,0] + [-2,0] + [-1,0]	\\
\hline
\end{spreadtab}

\vfill

% Preistabelle
\rowcolors{1}{}{}
\Large
\begin{tabular}{|l|r|}
	\hline
	\textbf{Preise} & \\ \hline
	\textbf{Wasser, Spezi, Orangenlimonade, Radler, Loschi Cola, Zitronenlimo} & \textbf{\EUR{0,50}} \\ \hline
	\textbf{Apfelschorle, ACE, Mate} & \textbf{\EUR{0,70}} \\ \hline
	\textbf{Bier} & \textbf{\EUR{0,80}} \\ \hline
\end{tabular}

\vfill

% Notizen
\normalsize
\textcolor{red}{Liste nur für Aktive!} Zum Aufladen, bezahlt in die \textbf{rote Kasse} und tragt es in der Spalte "'\textbf{Guthaben}"' ein. Bei Fragen bitte an Christopher - \texttt{christopher.sauer@fau.de} wenden. Mir ist die Zahlendarstellung bewusst!

\textbf{Stand: \today}
\end{center}

\end{document}
