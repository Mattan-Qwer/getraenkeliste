\documentclass[paper=a4, twoside=false, fontsize=16pt, headings=normal]{article} % DIN-A4 Formatierung
%-- Pakete --%
\usepackage[ngerman]{babel} %Ausgabe von Titeln in Deutsch
\usepackage[utf8]{inputenc} %Umlaute
% bessere Schriftarten
\usepackage{mathptmx, charter, courier}
\usepackage[scaled]{helvet}
\usepackage{lmodern}
\renewcommand*\familydefault{\sfdefault}
% Schriftverbesserung, microtype
\usepackage{microtype}
% farbige Schrift
\usepackage{color, colortbl}
\usepackage[table]{xcolor}
\definecolor{lightgray}{gray}{0.9}
% Bilder
\usepackage{graphicx}
% URLs
\usepackage{hyperref}
%-- Pagestyle --%
% Zeilenabstand
\usepackage{setspace}
% Zeichenabstand
\usepackage{xspace}
% Symbole
\usepackage{marvosym}
\usepackage{skull}
%-- Tabellen --%
\usepackage{spreadtab}
\usepackage{numprint}
\usepackage{eurosym}
\usepackage{fp}
\usepackage{longtable}
% Seitenränder
\usepackage[left=2cm,right=2cm,top=1cm,bottom=1cm]{geometry}
\usepackage{booktabs}
% Keine Seitenzahlen usw.
\pagestyle{empty}

\begin{document}
	
	% Formatierung
	%\onehalfspacing
	\begin{center}
		
		% Um Logo in Zeile vertikal zu zentrieren
		
		%\raisebox{-.05\height}{\includegraphics[width=0.15\textwidth]{logo}} 
		\Large \textbf{Getränkestrichliste} - \textbf{Stand: \today}
				% Formatierung - normale Schrift
		\large
		%\normalsize
		
		% Optionen
		%\renewcommand\STprintnum[1]{\FPifneg{#1}\color{red}\fi\numprint{#1}}
		\npdecimalsign{,}
		\npthousandsep{.}
		
		
		% hline-Dicke und Zellenfarbe
		\setlength{\arrayrulewidth}{.09em}
		
		% Getränkeliste
		\rowcolors{1}{}{lightgray}
		\begin{longtable}{|l|r|p{3.7cm}|p{3.7cm}|p{2.2cm}|c|}
			\hline 
            \textbf{Name}  & \textbf{Guthaben} & \textbf{\EUR{0,50}}	& \textbf{\EUR{0,70}} 	& \textbf{\EUR{1,00}} 	& \textbf{Score} 			\\
			\hline
%Hier kommen die Daten rein
			%
			\hline
		\end{longtable}
		
		\vspace{0.5cm}
		
		\large
		% Preistabelle
		\rowcolors{1}{}{}
		\begin{tabular}{|p{15cm}|r|}
			\hline
			\textbf{Was?} & \textbf{Preis} \\ \hline
			Alles was \EUR{0,50} kostet & \EUR{0,50} \\ \hline
			Alles was \EUR{0,70} kostet & \EUR{0,70} \\ \hline
			Alles was \EUR{1,00} kostet & \EUR{1,00} \\ \hline
		\end{tabular}
		
		\vfill
		
		% Notizen
		\Large{\textcolor{red}{Positiv-Liste nur für Aktive!}} Zum Aufladen, bezahlt in die \textbf{rote Kasse} und tragt es in der Spalte \glqq \textbf{Guthaben} \grqq ein. Bei Fragen bitte an \texttt{getraenke@fablab.fau.de} wenden.
		
		\vfill
		
		\Large \textbf{Liste für neue Betreuer} (werden beim nächsten Mal hinzugefügt)
		\large
		
		\rowcolors{1}{}{lightgray}
		\begin{tabular}{|p{2.4cm}|p{1.9cm}|p{3.7cm}|p{3.7cm}|p{2.2cm}|c|}
			\hline
			\textbf{Name} 		& \textbf{Guthaben}  									& \textbf{\EUR{0,50}}	& \textbf{\EUR{0,70}} 	& \textbf{\EUR{1,00}} 	& \textbf{Score} 			\\
			\hline
			%
			& & & & & \\
			\hline
			%
			& & & & & \\
			\hline
			%
			& & & & & \\
			\hline
		\end{tabular}
	\end{center}
	
\end{document}
